\begin{question}[section=13,name={Mean Effektive Gain},difficulty=,quantity=,type=thr,tags={20131024,20130724}]
	Was ist der Mean Effektive Gain und in welchem Zusammenhang wird er verwendet?
	
	%\\ \textbf{Hinweis:}\\
	
\end{question}
\begin{solution}
	Wenn mehrere Wellen auf die Antenne einfallen (z.B. im Mobilfunk), kann nicht mehr der ursprüngliche Gewinn für HEW verwendet werden. Die Gewinndefinition wird mit Wahrscheinlichkeitsverteilung p($\theta,\varphi$) der HEW über alle möglichen Empfangsrichtungen zu einem richtungsabhängigen Gewinn erweitert. MEG kommt vorallem im Mobilfunk zur Anwendung. Im Pegelplan wird der ideale Gewinn der Handy-Antenne durch den MEG ersetzt. Ein typischerweise liegt der Wert bei ca. -10dBi, welcher den Effekt der Mehrwegeausbreitung in Kombination mit der Handy-Antenne, Hand, Kopf und Schulter des Benützers vereint.
	\begin{equation}
		MEG= \int \limits_{4\pi} G(\vartheta,\varphi) p(\vartheta,\varphi) d\Omega
	\end{equation}
\end{solution}