\begin{question}[section=6,name={Grundmodus vergleichen},difficulty=,quantity=2,type=thr,tags={20131210}]
	Geben Sie den Grundmodus der Parallelplattenleitung, des Rechteckhohlwellenleiters und des Koaxialkabels an!
	\\ \textbf{Hinweis:}\\
	Skript Seite 68, 43, 61
\end{question}
\begin{solution}
	Als Grundmodus wird der Wellentyp mit der niedrigsten Grundfrequenz bezeichnet.\\
Die Grundfrequenz ist von den geometrischen Abmessungen und der Dielektrizität bzw. Permeabilität bestimmt.\\

Parallelplattenleitung: TEM-Welle, Grenzfrequenz ist \unit{0}\hertz.\\


Rechteckhohlwellenleiter: TE$_{10}$-Modus, auch als H$_{10}$-Modus bezeichnet.\\
\begin{align}
	\lambda_G &= 2 \cdot a \qquad\quad \\
	\lambda_H &= \frac{1}{\sqrt{1-\left(\frac{\lambda}{\lambda_G}\right)^2}}
\end{align}

Koaxialkabel: wie bei Parallelplattenleitung $\Rightarrow$ TEM-Welle, keine untere Grundfrequenz.
\end{solution}