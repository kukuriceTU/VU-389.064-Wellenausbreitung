\begin{question}[section=11,name={Anpassungsnetzwerk},difficulty=,quantity=1,type=thr,tags={}]
	Eine Antenne mit $4000 ~\Omega$  Fusspunktimpedanz soll mit einem Koaxialkabel von $50 ~\Omega$ Impedanz gespeist werden. Welche Aufgabe hat hierbei ein Anpassungsnetzwerk, und wo wäre es im Idealfall anzuordnen?
	\\ \textbf{Hinweis:}\\
	
\end{question}
\begin{solution}
	Anzuordnen: Zwischen Antenne und Leitung\\
	Das Anpassungsnetzwerk wirkt als Impedanzwandler der auf der einen seite $50~\Omega$ und auf der Anderen $4000~\Omega$ Wellenwiderstand besitzt. Anpassung für maximale Wirkleistungsübertragung $Z_G' =Z_A^*$
	
\end{solution}