\begin{question}[section=3,name={Depolarisation},difficulty=,quantity=5,type=thr,tags={20160310,20130314}]
	Beschreiben sie stichwortartig drei Depolarisationsmechanismen bei der Funkübertragung!
	\\ \textbf{Hinweis:}\\
	Skript Seite 33
\end{question}
\begin{solution}
	\begin{itemize}
		\item{Reflexion: der Reflexionskoeffizient ist für TE und TM-Fall unterschiedlich. Daher werden die Polarisationseigenschaften geändert.}
		\item{Induzierte Ströme bei Antennen: in leitfähigen Elementen (nicht primär Teil der Antenne, z.\,B.: Befestigungselemente), können als Sekundärstrahler wirken.}
		\item{Bei Antenne knapp über dem Erdboden können Ströme im Erdboden induziert werden, die als Sekundärstrahler wirken (siehe Spiegelungsprinzip der Elektrostatik).}
		\item{Medien mit unterschiedlicher Dämpfung für horizontale und vertikale Polarisation (z.\,B.: Wälder dämpfen vertikal polarisierte Felder stärker als horizontal polarisierte.}
		\item{Beugung: In der unmittelbaren Nähe von beugenden Kanten sind polarisationsabhängige Beugungseffekte zu beobachten.}
		\item{In der Ionosphäre erfolgt eine Drehung der Polarisationsrichtung. Faradayeffekt}
	\end{itemize}
\end{solution}