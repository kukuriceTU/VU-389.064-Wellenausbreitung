\begin{question}[section=8,name={Übertragungsgrenzen},difficulty=,quantity=4,type=thr,tags={20131024}]
	Erklären Sie die Unterschiede zwischen Dispersionsbegrenzung und Dämpfungsbegrenzung bei Nachrichtenübertragung über Wellenleiter!
	\\ \textbf{Hinweis:}\\
	
\end{question}
\begin{solution}
	Dispersionsbegrenzung: Begrenzung durch Verzerrung der Signale. Es bezeichnet auch das breiter werden von schmalen Pulsen und schlussendlichem ineinander laufen. (ISI)\\
	Dämpfungsbegrenzung: Begrenzung durch Signalstärke, Signal geht im Rauschen unter.
\end{solution}